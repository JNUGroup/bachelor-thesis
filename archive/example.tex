\begin{table}
        \centering
        \begin{tabular}{cccp{38mm}}
            \toprule
            \textbf{文档域类型} & \textbf{Java类型} & \textbf{宽度(字节)} & \textbf{说明} \\
            \midrule
            BOOLEAN  & boolean &  1  & \\
            CHAR     & char    &  2  & UTF-16字符 \\
            BYTE     & byte    &  1  & 有符号8位整数 \\
            SHORT    & short   &  2  & 有符号16位整数 \\
            INT      & int     &  4  & 有符号32位整数 \\
            LONG     & long    &  8  & 有符号64位整数 \\
            STRING   & String  &  字符串长度  & 以UTF-8编码存储 \\
            DATE     & java.util.Date & 8 & 距离GMT时间1970年1月1日0点0分0秒的毫秒数 \\
            BYTE\_ARRAY & byte$[]$ & 数组长度 & 用于存储二进制值 \\
            BIG\_INTEGER & java.math.BigInteger & 和具体值有关 & 任意精度的长整数 \\
            BIG\_DECIMAL & java.math.BigDecimal & 和具体值有关 & 任意精度的十进制实数 \\
            \bottomrule
        \end{tabular}
        \caption{测试表格}\label{table:test1}
    \end{table}



测试一下脚注\footnote{测试脚注},测试一下脚注\footnote{测试脚注},测试一下脚
注\footnote{测试脚注},测试一下脚注\footnote{测试脚注},测试一下脚注\footnote{测
    试脚注},测试一下脚注\footnote{测试脚注},测试一下脚注\footnote{测试脚注},测
试一下脚注\footnote{测试脚注},测试一下脚注\footnote{测试脚注},测试一下脚
注\footnote{测试脚注}。

测试一下引用\cite{newman2006structure},连续引用
\cite{newman2001random,aiello2000random,bollobas2001random},另一个连续引用
\cite{newman2001random,bollobas2001random,barabasi1999emergence}。测试一下带页码
的引用\cite[124--128]{erdHos1961strength}。

下面是一个项目列表:

\begin{itemize}
    \item 这是第一项。这是第一项。
    \item 这是第二项。这是第二项。这是第二项。这是第二项。这是第二项。这是第二项。这
    是第二项。这是第二项。这是第二项。这是第二项。这是第二项。
    \item 这是第三项。这是第三项。这是第三项。
    \begin{itemize}
        \item 测试第二层列表。测试第二层列表。
        \item 测试第二层列表。测试第二层列表。
        \begin{itemize}
            \item 测试第三层列表。测试第三层列表。
            \item 测试第三层列表。测试第三层列表。
        \end{itemize}
        \item 测试第二层列表。测试第二层列表。测试第二层列表。测试第二层列表。测试第二
        层列表。
    \end{itemize}
    \item 这是第四项。这是第四项。这是第四项。
    \begin{enumerate}
        \item 测试第二层列表。测试第二层列表。测试第二层列表。测试第二层列表。测试第二
        层列表。测试第二层列表。测试第二层列表。测试第二层列表。
        \item 测试第二层列表。测试第二层列表。
        \item 测试第二层列表。测试第二层列表。测试第二层列表。测试第二层列表。测试第二
        层列表。
    \end{enumerate}
\end{itemize}

下面是一个编号列表:

\begin{enumerate}
    \item 这是第一项。这是第一项。这是第一项。这是第一项。这是第一项。这是第一项。这
    是第一项。这是第一项。这是第一项。这是第一项。这是第一项。
    \item 这是第二项。这是第二项。
    \item 这是第三项。这是第三项。这是第三项。
    \begin{itemize}
        \item 测试第二层列表。测试第二层列表。
        \item 测试第二层列表。测试第二层列表。
        \item 测试第二层列表。测试第二层列表。测试第二层列表。测试第二层列表。测试第二
        层列表。
    \end{itemize}
    \item 这是第四项。这是第四项。这是第四项。
    \begin{enumerate}
        \item 测试第二层列表。测试第二层列表。
        \begin{enumerate}
            \item 测试第三层列表。测试第三层列表。测试第三层列表。测试第三层列表。测试第三
            层列表。测试第三层列表。
            \item 测试第三层列表。测试第三层列表。
            \item 测试第三层列表。测试第三层列表。测试第三层列表。
        \end{enumerate}
        \item 测试第二层列表。测试第二层列表。测试第二层列表。
    \end{enumerate}
\end{enumerate}

下面是最多三层的阿拉伯数字列表:
\begin{arabicenum}
    \item 第1项
    \item 第2项
    \begin{arabicenum}
        \item 第2.1项
        \item 第2.2项
        \begin{arabicenum}
            \item 第2.2.1项
            \item 第2.2.2项
            \item 第2.2.3项
        \end{arabicenum}
        \item 第2.3项
    \end{arabicenum}
    \item 第3项
\end{arabicenum}

下面是最多两层的罗马数字列表:
\begin{romanenum}
    \item 第1项
    \item 第2项
    \begin{romanenum}
        \item 第2.1项
        \item 第2.2项
        \item 第2.3项
    \end{romanenum}
    \item 第3项
\end{romanenum}

下面是最多两层的小写字母列表:
\begin{alphaenum}
    \item 第1项
    \item 第2项
    \begin{alphaenum}
        \item 第2.1项
        \item 第2.2项
        \item 第2.3项
    \end{alphaenum}
    \item 第3项
\end{alphaenum}

下面是最多两层的情况列表:
\begin{caseenum}
    \item 第1项
    \item 第2项
    \begin{caseenum}
        \item 第2.1项
        \item 第2.2项
        \item 第2.3项
    \end{caseenum}
    \item 第3项
\end{caseenum}

下面是最多两层的步骤列表:
\begin{stepenum}
    \item 第1项
    \item 第2项
    \begin{stepenum}
        \item 第2.1项
        \item 第2.2项
        \item 第2.3项
    \end{stepenum}
    \item 第3项
\end{stepenum}

下面测试一下引用环境|quote|。下面测试一下引用环境|quote|。下面测试一下引用环境|quote|。
下面测试一下引用环境|quote|。下面测试一下引用环境|quote|。下面测试一下引用环境|quote|。
下面测试一下引用环境|quote|。下面测试一下引用环境|quote|。下面测试一下引用环境|quote|。

\begin{quote}
    这是一段引用。这是一段引用。这是一段引用。这是一段引用。这是一段引用。这是一段引用。
    这是一段引用。这是一段引用。这是一段引用。这是一段引用。这是一段引用。这是一段引用。
    这是一段引用。这是一段引用。这是一段引用。这是一段引用。这是一段引用。
    
    这是一段引用。这是一段引用。这是一段引用。这是一段引用。这是一段引用。这是一段引用。
    这是一段引用。这是一段引用。这是一段引用。
    
    这是一段引用。这是一段引用。
\end{quote}

下面测试一下引用环境|quotation|。下面测试一下引用环境|quotation|。
下面测试一下引用环境|quotation|。下面测试一下引用环境|quotation|。
下面测试一下引用环境|quotation|。下面测试一下引用环境|quotation|。
下面测试一下引用环境|quotation|。下面测试一下引用环境|quotation|。
下面测试一下引用环境|quotation|。

\begin{quotation}
    这是一段引用。这是一段引用。这是一段引用。这是一段引用。这是一段引用。这是一段引用。
    这是一段引用。这是一段引用。这是一段引用。这是一段引用。这是一段引用。这是一段引用。
    这是一段引用。这是一段引用。这是一段引用。这是一段引用。这是一段引用。
    
    这是一段引用。这是一段引用。这是一段引用。这是一段引用。这是一段引用。这是一段引用。
    这是一段引用。这是一段引用。这是一段引用。
    
    这是一段引用。这是一段引用。
\end{quotation}

引用结束。引用结束。引用结束。引用结束。引用结束。引用结束。引用结束。引用结束。引用结束。
引用结束。引用结束。引用结束。

测试一下定理环境。

\begin{theorem}[测试定理]
    测试一下定理环境。测试一下定理环境。测试一下定理环境。测试一下定理环境。测试一下
    定理环境。测试一下定理环境。测试一下定理环境。
\end{theorem}
\begin{proof}
    \blindtext
\end{proof}

\blindtext

\begin{theorem}
    测试一下定理环境。测试一下定理环境。测试一下定理环境。测试一下定理环境。测试一下
    定理环境。测试一下定理环境。测试一下定理环境。
\end{theorem}
\begin{proof}
    \blindtext
\end{proof}

\blindtext

\begin{lemma}
    测试一下定理环境。测试一下定理环境。测试一下定理环境。测试一下定理环境。测试一下
    定理环境。测试一下定理环境。测试一下定理环境。
\end{lemma}
\begin{proof}
    \blindtext
\end{proof}

\blindtext

\begin{definition}
    测试一下定理环境。测试一下定理环境。测试一下定理环境。测试一下定理环境。测试一下
    定理环境。测试一下定理环境。测试一下定理环境。
\end{definition}

\blindtext

\begin{corollary}
    测试一下定理环境。测试一下定理环境。测试一下定理环境。测试一下定理环境。测试一下
    定理环境。测试一下定理环境。测试一下定理环境。
\end{corollary}

\blindtext

\begin{proposition}
    测试一下定理环境。测试一下定理环境。测试一下定理环境。测试一下定理环境。测试一下
    定理环境。测试一下定理环境。测试一下定理环境。
\end{proposition}

\blindtext

\begin{fact}
    测试一下定理环境。测试一下定理环境。测试一下定理环境。测试一下定理环境。测试一下
    定理环境。测试一下定理环境。测试一下定理环境。
\end{fact}

\blindtext

\begin{assumption}
    测试一下定理环境。测试一下定理环境。测试一下定理环境。测试一下定理环境。测试一下
    定理环境。测试一下定理环境。测试一下定理环境。
\end{assumption}

\blindtext

\begin{conjecture}
    测试一下定理环境。测试一下定理环境。测试一下定理环境。测试一下定理环境。测试一下
    定理环境。测试一下定理环境。测试一下定理环境。
\end{conjecture}

\blindtext

\begin{hypothesis}
    测试一下定理环境。测试一下定理环境。测试一下定理环境。测试一下定理环境。测试一下
    定理环境。测试一下定理环境。测试一下定理环境。
\end{hypothesis}

\blindtext

\begin{axiom}
    测试一下定理环境。测试一下定理环境。测试一下定理环境。测试一下定理环境。测试一下
    定理环境。测试一下定理环境。测试一下定理环境。
\end{axiom}

\blindtext

\begin{postulate}
    测试一下定理环境。测试一下定理环境。测试一下定理环境。测试一下定理环境。测试一下
    定理环境。测试一下定理环境。测试一下定理环境。
\end{postulate}

\blindtext

\begin{principle}
    测试一下定理环境。测试一下定理环境。测试一下定理环境。测试一下定理环境。测试一下
    定理环境。测试一下定理环境。测试一下定理环境。
\end{principle}

\blindtext

\begin{problem}
    测试一下定理环境。测试一下定理环境。测试一下定理环境。测试一下定理环境。测试一下
    定理环境。测试一下定理环境。测试一下定理环境。
\end{problem}
\begin{solution}
    \blindtext
\end{solution}

\blindtext

\begin{problem}
    测试一下定理环境。测试一下定理环境。测试一下定理环境。测试一下定理环境。测试一下
    定理环境。测试一下定理环境。测试一下定理环境。
\end{problem}
\begin{solution}
    \blindtext
\end{solution}

\blindtext

\begin{exercise}
    测试一下定理环境。测试一下定理环境。测试一下定理环境。测试一下定理环境。测试一下
    定理环境。测试一下定理环境。测试一下定理环境。
\end{exercise}

\blindtext

\begin{exercise}
    测试一下定理环境。测试一下定理环境。测试一下定理环境。测试一下定理环境。测试一下
    定理环境。测试一下定理环境。测试一下定理环境。
\end{exercise}

\begin{algorithm}
    测试一下定理环境。测试一下定理环境。测试一下定理环境。测试一下定理环境。测试一下
    定理环境。测试一下定理环境。测试一下定理环境。
\end{algorithm}

\subsection{中心观点与思想}

云计算在概念上通常被分为IaaS、PaaS、SaaS几个层面。但透过分类去理解其本
质,可认为是上世纪70年代基于大型计算机的中心控制型瘦客户端终端模式,在
如今技术水平上的一种新的表达,是在技术发展道路中,螺旋上升的结果。

与瘦客户端相比,云计算在设计结构上存在一定的相似性。

\begin{enumerate}
    \item 中心控制的模式:通过中心的大规模硬件提供统一的计算,可大大降低管理成本,提
    高硬件资源利用率,同时降低客户端的硬件成本需求。例如Nvidia推出Georce GRID平台
    \cite{NVIDIAGRID},推出了GaaS\footnote{Gaming as a Service}概念。将
    GPU放置在云端,使得用户不需要再不断购买升级显卡,并可在更为广泛的终端(包括手机、
    平板、智能电视)和地点体验最新的游戏。
    \item 数据集中:由于瘦客户端的关系,数据都集中存储在中心,可对数据提供
    可靠的保护,并且通过按需调用的实现方式,降低对网络带宽的需求。
\end{enumerate}




测试一下中文字体:

{\songti\zihao{0} 宋体,初号}

{\songti\zihao{-0} 宋体,小初}

{\songti\zihao{1} 宋体,一号}

{\songti\zihao{-1} 宋体,小一}

{\songti\zihao{2} 宋体,二号}

{\songti\zihao{-2} 宋体,小二}

{\songti\zihao{3} 宋体,三号}

{\songti\zihao{-3} 宋体,小三}

{\songti\zihao{4} 宋体,四号}

{\songti\zihao{-4} 宋体,小四}

{\songti\zihao{5} 宋体,五号}

{\songti\zihao{-5} 宋体,小五}

{\songti\zihao{6} 宋体,六号}

{\songti\zihao{-6} 宋体,小六}

{\songti\zihao{7} 宋体,七号}

{\songti\zihao{8} 宋体,八号}

{\heiti\zihao{0} 黑体,初号}

{\heiti\zihao{-0} 黑体,小初}

{\heiti\zihao{1} 黑体,一号}

{\heiti\zihao{-1} 黑体,小一}

{\heiti\zihao{2} 黑体,二号}

{\heiti\zihao{-2} 黑体,小二}

{\heiti\zihao{3} 黑体,三号}

{\heiti\zihao{-3} 黑体,小三}

{\heiti\zihao{4} 黑体,四号}

{\heiti\zihao{-4} 黑体,小四}

{\heiti\zihao{5} 黑体,五号}

{\heiti\zihao{-5} 黑体,小五}

{\heiti\zihao{6} 黑体,六号}

{\heiti\zihao{-6} 黑体,小六}

{\heiti\zihao{7} 黑体,七号}

{\heiti\zihao{8} 黑体,八号}

{\kaishu\zihao{0} 楷书,初号}

{\kaishu\zihao{-0} 楷书,小初}

{\kaishu\zihao{1} 楷书,一号}

{\kaishu\zihao{-1} 楷书,小一}

{\kaishu\zihao{2} 楷书,二号}

{\kaishu\zihao{-2} 楷书,小二}

{\kaishu\zihao{3} 楷书,三号}

{\kaishu\zihao{-3} 楷书,小三}

{\kaishu\zihao{4} 楷书,四号}

{\kaishu\zihao{-4} 楷书,小四}

{\kaishu\zihao{5} 楷书,五号}

{\kaishu\zihao{-5} 楷书,小五}

{\kaishu\zihao{6} 楷书,六号}

{\kaishu\zihao{-6} 楷书,小六}

{\kaishu\zihao{7} 楷书,七号}

{\kaishu\zihao{8} 楷书,八号}

{\fangsong\zihao{0} 仿宋,初号}

{\fangsong\zihao{-0} 仿宋,小初}

{\fangsong\zihao{1} 仿宋,一号}

{\fangsong\zihao{-1} 仿宋,小一}

{\fangsong\zihao{2} 仿宋,二号}

{\fangsong\zihao{-2} 仿宋,小二}

{\fangsong\zihao{3} 仿宋,三号}

{\fangsong\zihao{-3} 仿宋,小三}

{\fangsong\zihao{4} 仿宋,四号}

{\fangsong\zihao{-4} 仿宋,小四}

{\fangsong\zihao{5} 仿宋,五号}

{\fangsong\zihao{-5} 仿宋,小五}

{\fangsong\zihao{6} 仿宋,六号}

{\fangsong\zihao{-6} 仿宋,小六}

{\fangsong\zihao{7} 仿宋,七号}

{\fangsong\zihao{8} 仿宋,八号}

测试一下标准字号:

{\Huge 汉字,Huge}

{\huge 汉字,huge}

{\LARGE 汉字,LARGE}

{\Large 汉字,Large}

{\large 汉字,large}

{\normalsize 汉字,normalsize}

{\small 汉字,small}

{\footnotesize 汉字,footnotesize}

{\scriptsize 汉字,scriptsize}

{\tiny 汉字,tiny}

测试一下标准字体的变形:

{\songti 宋体} {\heiti 黑体} {\kaishu 楷书} {\fangsong 仿宋}

{\textsl{textsl字体}}

{\bfseries bfseries字体}

{\textbf{textbf字体}}

{\textit{textit字体}}

测试一下数学公式中的字体大小。

\newcommand{\set}[1]{\left\{\,#1\,\right\}}
\newcommand{\card}[1]{\left|\,#1\,\right|}

Fall-Out指标计算公式如下:
\begin{equation*}
\mbox{fallout} = \frac{\card{\set{\text{不相关文档}}\cap\set{\text{获取的文档}}}}{\card{\set{\text{不相关文档}}}}
\end{equation*}